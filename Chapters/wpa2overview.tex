\part{Theoretical Knowledge}
\chapter{Overview of WPA2 and the Four-Way Handshake}
Wi-Fi Protected Access II (WPA2) is the second generation of the WPA security protocol, designed to secure wireless networks. Based on the IEEE 802.11i standard, WPA2 became mandatory for all Wi-Fi-certified devices starting in 2006. This protocol addresses significant vulnerabilities found in its predecessor, WPA, and introduces critical security enhancements that ensure robust encryption and data integrity.

\section{Key Features of WPA2}
WPA2 offers several key improvements over WPA, the most notable of which are enhanced encryption, improved data integrity, and mandatory certification. These advancements make WPA2 a trusted solution for both enterprise and personal wireless network security.

\begin{itemize}
    \item \textbf{Enhanced Encryption}: WPA2 replaces WPA's Temporal Key Integrity Protocol (TKIP) with the Counter Mode Cipher Block Chaining Message Authentication Code Protocol (CCMP). CCMP leverages the Advanced Encryption Standard (AES), a much stronger encryption algorithm that uses a 128-bit key. AES ensures higher levels of confidentiality and integrity for transmitted data, providing resistance against brute-force attacks and cryptographic weaknesses.
    
    \item \textbf{Improved Data Integrity}: WPA2 employs CCMP for message integrity, replacing the weaker Message Integrity Code (MIC) used in WPA. CCMP ensures that data cannot be altered or tampered with during transmission, thereby preventing attacks such as packet injection or replay attacks. The shift to AES-based encryption and a more robust integrity mechanism enhances overall network security.
    
    \item \textbf{Mandatory Certification and Testing}: WPA2 certification, enforced by the Wi-Fi Alliance, ensures that devices meet the rigorous security standards required for Wi-Fi networks. Since March 2006, all Wi-Fi-certified devices have been required to support WPA2, making it the default security protocol for most wireless devices.
\end{itemize}

\section{How WPA2 Works: The Four-Way Handshake}
The primary method by which WPA2 ensures secure communication between clients (stations) and access points (APs) is through the Four-Way Handshake. This handshake is essential for establishing a secure communication session by verifying the legitimacy of both parties and generating the necessary encryption keys.

The Four-Way Handshake is designed to prevent eavesdropping, man-in-the-middle attacks, and replay attacks, and to ensure the integrity of the encryption keys used in communication. Below is a detailed explanation of the handshake process:

\subsection{Key Concepts}
To understand the Four-Way Handshake, it is important to first define several key cryptographic elements involved in the process:

\begin{itemize}
    \item \textbf{PMK (Pairwise Master Key)}: The PMK is derived from the pre-shared key (PSK) or, in enterprise networks, from an 802.1X authentication server. It is a 256-bit key that serves as the root key for generating other session keys used in the handshake process.
    
    \item \textbf{PTK (Pairwise Transient Key)}: The PTK is the key used to encrypt unicast traffic between the client and the AP. It is derived from the PMK, as well as random values (nonces) generated by both parties.
    
    \item \textbf{GTK (Group Temporal Key)}: The GTK is used to encrypt multicast and broadcast traffic between the AP and its clients. It is generated by the AP and shared with all clients connected to the network.
    
    \item \textbf{ANONCE and SNONCE}: These are random nonces generated by the AP and client, respectively. They play a critical role in ensuring the freshness of the session and preventing replay attacks.
    
    \item \textbf{MIC (Message Integrity Code)}: The MIC is a cryptographic checksum used to ensure the integrity of the messages exchanged during the handshake. It prevents tampering and assures both parties that the data has not been modified.
\end{itemize}

\subsection{The Four-Way Handshake Process}
The Four-Way Handshake consists of four messages exchanged between the client and the AP, during which the cryptographic keys necessary for securing the connection are derived and exchanged. The process can be broken down as follows:

\begin{enumerate}
    \item \textbf{Message 1 (AP to Client)}: The AP sends a message containing its nonce (ANONCE) to the client. This nonce is required by the client to generate the Pairwise Transient Key (PTK). At this point, the client already has the pre-shared key (PSK), which is used to derive the Pairwise Master Key (PMK).
    
    \item \textbf{Message 2 (Client to AP)}: The client generates its own nonce (SNONCE) and sends it to the AP, along with a Message Integrity Code (MIC). The MIC is a cryptographic signature that allows the AP to verify that the message is legitimate and originated from the client. Upon receiving the message, the AP derives the PTK, using the PMK, ANONCE, SNONCE, and the MAC addresses of both the AP and the client.
    
    \item \textbf{Message 3 (AP to Client)}: The AP sends the Group Temporal Key (GTK) to the client. The GTK is used for encrypting multicast and broadcast traffic, ensuring that all clients on the same AP can securely communicate. The client installs the GTK, which allows it to handle group traffic.
    
    \item \textbf{Message 4 (Client to AP)}: The client sends a confirmation message to the AP, indicating that it has successfully installed both the PTK and GTK. This final step completes the handshake and secures the communication channel between the client and the AP.
\end{enumerate}

The exchange of these four messages establishes a secure session, where the PTK is used for unicast traffic encryption, and the GTK is used for broadcast and multicast traffic encryption.

\section{Security Implications and Cracking WPA2}
The WPA2 Four-Way Handshake plays a crucial role in maintaining the security of wireless networks. However, the handshake's reliance on the pre-shared key (PSK) means that the strength of the security depends heavily on the strength of the password chosen by the network administrator. 

An attacker can exploit the handshake process by capturing the messages exchanged between the client and the AP. One common method is to force a client to disconnect and reconnect, thereby triggering a new handshake. Once captured, the attacker can attempt to crack the pre-shared key (PSK) by performing a brute-force attack using tools like \textbf{aircrack-ng}.




